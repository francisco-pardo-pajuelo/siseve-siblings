
\begin{abstract}

Family and social networks play a critical role in shaping educational decisions, with sibling influence being particularly significant. This paper examines the spillover effects of older siblings’ college admission on the educational trajectories of their younger siblings, from school performance and completion to college application decisions and outcomes during both the admissions process and college attendance. I leverage admission cutoffs in Peru’s decentralized public college system, where each institution administers its own entrance exam and application process, to isolate exogenous variation in college entry. The results show that younger siblings improve their academic performance in school and are significantly more likely to apply to four-year colleges when an older sibling is admitted. Using complementary survey data, I find that increased parental expectations are a key channel driving these effects. These findings suggest that in environments where college access is limited and admission processes are complex, siblings play an especially important role in bridging information gaps and serving as aspirational role models.
\\
\textit{JEL Codes: I21, I24}
\end{abstract}

%   I2 Education and Research Institutions
    % 	I20 	General 
    %   I21 	Analysis of Education
    %	I23 	Higher Education • Research Institutions
    %	I24 	Education and Inequality
    %	I26 	Returns to Education 
    %   I28 	Government Policy     
    
%   D1 	Household Behavior and Family Economics 
    % 	D13 	Household Production and Intrahousehold Allocation 
    %   D19  other

%   J2 	Demand and Supply of Labor 
    %   J24 	Human Capital • Skills • Occupational Choice • Labor Productivity 

%   O1 	Economic Development 
    %   O15 	Human Resources • Human Development • Income Distribution • Migration 

%   R2 	Household Analysis  
    % 	R23 	Regional Migration • Regional Labor Markets • Population • Neighborhood Characteristics 





